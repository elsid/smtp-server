\chapter{Технологический раздел}
\label{cha:implementation}

\section{Работа сервера с хранилищем данных формата Maildir}

Для хранения полученных писем используется формат Maildir.
Для каждого домена получателя создается директория, в котрой создается директория пользователя в этом домене: <домен>/<пользователь>/Maildir.
В директории Maildir создаются директории tmp, new, cur.
После получения команды DATA сервер генерирует название файла: <время в секундах>\_<микросекунды>\_<pid>\_<случайное число>.eml.
Файл создается в директории получателя, почтовый адрес которого получен первым, tmp.
В файл записыватеся заголовок Return-path, содержащий адрес отправителя, и Received, содержащий домен клиента, домен сервера, протокол (SMTP), адрес первого получателя, дату и время на момент создания файла.
Все полученные данные записываются в файл.
После получения маркера конца данных файл закрывается и перемещается в директорию new системным вызовом rename.
Если транзакцию не удалось завершить, файл удаляется.
Для каждого пользователя, начиная с второго, в директории new создается жесткая ссылка на записанный файл системным вызовом link.
Клиент не использует директорию tmp, а сервером гарантируется создание уникальных имен файлов, поэтому не может возникнуть ситуации одновременно обращения к файлу в этой директории несколькими процессами или потоками.
Системный вызов rename атомарно перемещает файл без копирования, если не произошло сбоя.

\section{Сценарий сборки проекта}

Сборка проекта состоит из двух частей: сборка программы и сборка РПЗ.
Эти части выделены в отдельные файлы.
Сборка программы включает сборку РПЗ.

\lstinputlisting[caption=Сценарий сборки программы,label=lst:Makefile]{../Makefile}
\lstinputlisting[caption=Сценарий сборки РПЗ,label=lst:Makefile-doc]{Makefile}

\section{Файл конфигурации сервера}

Для настройки работы сервера используется файл конфигурации.
Для чтения файла используется библиотека libconfig.

\lstinputlisting[caption=Пример конфигурационного файла,label=lst:config]{../etc/smtp-server.cfg}

\begin{itemize}
\item \verb;address; -- адрес, который слушает сервер
\item \verb;port; -- порт, который слушает сервер
\item \verb;workers_count; -- количество процессов, обслуживающих соединение с клиентом
\item \verb;backlog_size; -- размер очереди ожидания приема соединения
\item \verb;maildir; -- путь к директории хранилища писем
\item \verb;log; -- путь к файлу журнала
\item \verb;max_message_in_size; -- размер буфера для принимаемых сообщений
\item \verb;timeout; -- таймаут
\item \verb;daemon; -- флаг необходимости демонизации процесса
\end{itemize}

\section{Тестирование}

\subsection{Платформа для тестирования}

Использовалась операционная система Ubuntu 14.04 LTS 64-bit.
Сборка осуществлялась компилятором clang.

\subsection{Модульное тестирование}

Для модульного тестирования использовалась бибилиотека CUnit.
Тестировался модуль для разбора команд с параметрами: EHLO, HELO, MAIL, RCPT.
Результаты выполнения тестов представлен в листинге~\ref{lst:test-module-log}.

\lstinputlisting[caption=Результат модульного тестирования,label=lst:test-module-log]{../var/log/test_module_result.log}

\subsection{Системное тестирование}

Системное тестирование выполнялось с помощью скрипта на языке Python.
Использовались модули smtplib, unittest, hamcrest.
Выполнялось позитивное и негативное тестирование.
Рассмотрены следующие сценарии:

\begin{itemize}
\item Проверка правильности восприятия корректных последовательностей комманд.
\item Проверка ошибочных ответов на некорректные последовательности комманд.
\item Проверка выполнения почтовой транзакции и их повторения в рамках одной сессии.
\item Проверка ошибочных ответов на команды с некорректным синтаксисом.
\item Проверка ошибочных ответов на команды, не поддерживаемые сервером.
\item Проверка ошибочных ответов на команды, не являющиеся частью протокола SMTP.
\item Проверка таймаута.
\item Проверка возможности восприятия команд с лишними пробельными символами.
\end{itemize}

Используемая конфигурация сервера представлена в листинге~\ref{lst:test-system-cfg}.

\lstinputlisting[caption=Конфигурация сервера для системного тестирования,label=lst:test-system-cfg]{../etc/test_system.cfg}

Результат работы скрипта для системного тестирования представлен в листинге~\ref{lst:test-system-log}.

\lstinputlisting[caption=Результат системного тестирования,label=lst:test-system-log]{../var/log/test_system_result.log}

\subsection{Поиск утечек памяти}

Для поиска утечек памяти использовался valgrind.
Запущенная с помощью него программа подвергалась нагрузочному тестированию.
Использовались следующие сценарии нагрузочного тестирования:
\begin{enumerate}
\item Mножество параллельных соединений, в каждом из который в рамках одной сессии выполнялось множество почтовых транзакций.
\item Mножество параллельных соединений, в каждом из который в рамках одной сессии выполнялась одна почтовая транзакция.
\item Одновременно первый и второй сценарии.
\end{enumerate}

В рамках одной почтовой транзакции выполнялась отправка письма в формате eml размером 20КБ, содержащим 294 строки текста.
Для реализации каждого сценария был написан скрипт на языке Python.
Используемая конфигурация сервера представлена в листинге~\ref{lst:test-memory-cfg}.

\lstinputlisting[caption=Конфигурация сервера для поиска утечек памяти,label=lst:test-memory-cfg]{../etc/test_memory.cfg}

В ходе тестирования были выявлены утечки, после устраннения которых получен следующий результат представленный в листинге~\ref{lst:test-memory-log}.

\lstinputlisting[caption=Результат выполнения программы под управлением valgrind, label=lst:test-memory-log]{../var/log/test_memory_result.log}
